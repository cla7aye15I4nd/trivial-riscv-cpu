\documentclass[UTF8]{ctexart}
\usepackage{graphicx}
\usepackage[colorlinks,linkcolor=blue]{hyperref}

\CTEXsetup[format={\Large\bfseries}]{section}

\begin{document}

\title{AIRA RISC-V CPU Manual}
\author{于峥 518030910437}
\maketitle

This project is a trivial RISC-V CPU with tomasulo algorithm implemented in Verilog HDL, which is a course project of Computer Architecture, ACM Class @ SJTU.
\section{Design \& Architecture}
    \subsection{Environment}
    \begin{center}
        \begin{tabular}{c|c}
            \hline
            Device Name & Aira \\
            \hline
            ISA & RISCV 32I \\
            \hline
            FPGA & Basys3 \\
            \hline
        \end{tabular}
    \end{center}
    \subsection{Out-of-order Execution}
    The main feature of the Aira RISC-V is to support out-of-order execution, some algorithm used in the 
    Aira RISC-V are brefly introduced in the table below:
    \begin{center}
        \begin{tabular}{c|c}
            \hline
            Features & RISC-V CPU\\
            \hline
            Dynamic Scheduling & To\includegraphics[scale=0.1]{horse.png}sulo Algorithm \\
            \hline
            Piplining & 3-stages pipeline (Fetch, Dispatch, Execute)\\
            \hline
            Multiple FU & 2 Arithmetic/Logic Units and 1 Load/Store Unit \\
            \hline
        \end{tabular}
    \end{center}

    \subsubsection{Details}
    \begin{itemize}
    \item In order to achieve fetch one instruction four cycles, every cycle the fetcher will send pc and pc+4 to the I-cache.
    \item The fetcher receive instruction at negedge.
    \item For JAL instruction, fetcher calculate the target address and modify pc directly.
    \item Allocator rename register at dispatch stage.
    \item Every EX unit has a reserved station.
    \end{itemize}
    \subsubsection{Design Diagram}

    \begin{figure}[h]
    \includegraphics[scale=0.32]{DesignDiagram.jpg}
    \caption{Aira RISC-V CPU design diagram}
    \end{figure}

    \subsection{Cache}
    \begin{center}
        \begin{tabular}{ll}
            \hline
            Parameter & Data\\
            \hline
            Type & Instruction Cache \\
            \hline
            Size & 512B              \\
            \hline
            Set Associative & 2-way associative \\
            \hline
            Replacement Policy & FIFO \\
            \hline
        \end{tabular}
    \end{center}
    \subsection{Other Features}
    Using LED light for output and debug, and segment display as timer.
\section{Performance}
    The highest frequency that the CPU can reach is 200MHz, here are time consuming of some test points:
    \begin{center}
        \begin{tabular}{cc}
            \hline
            testcase & time(s)\\
            \hline
            pi & 1.15 \\
            \hline
            bulgarian & 3.76 \\
            \hline
            hanoi & 2.78 \\
            \hline
            queens & 1.03 \\
            \hline
        \end{tabular}
    \end{center}
    from the waveform, we can see multiple ex unit works at same time.
    
\section{References}
    \begin{itemize}
        \item [1.] 
        John L. Hennessy, David A. Patterson, et al. Computer Architecture: A Quantitative Approach, Fifth Edition, 2012.
        \item [2.]
        雷思磊. 自己动手写 CPU. 电子工业出版社, 2014.
        \item [3.]
        \href{http://riscv.org/specifications/}{RISC-V ISA Specification}
    \end{itemize}
\end{document}
